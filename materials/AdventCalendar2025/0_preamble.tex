%==========
%LuaLaTeXで実行する!!
% \documentclass[
% 		book,
% 		head_space=20mm,
% 		foot_space=20mm,
% 		gutter=10mm,
% 		line_length=190mm
% ]{jlreq}
%----------
%----------
%%%%% bookの設定
%  \NewPageStyle{mystyle}{
% 	yoko,
% 	running_head_position={top-center},
% 	nombre_position={top-fore-edge},
% 	odd_running_head={_section},
% 	even_running_head={_chapter}
% 	}
%  \pagestyle{mystyle}
%----------
\usepackage{luatexja}
%==========
%----------
%目次のレイアウト調整
\setcounter{tocdepth}{4}
%-----
\makeatletter
\def\l@part{%
  % |\@dottedtocline|の代わりに独自の命令を使う
  \my@dottedtocline
  {0}					%見出しのレベル
  {1.0\baselineskip}		%前の行とのアキ
  {0pt}					%字下げ
  {6\zw}				%ラベル(「第1章」等)の幅
%   {\sffamily\bfseries\gtfamily\large}	%フォント
  {\Large}				%フォント
%   {\rmfamily}			%フォント
}
\def\l@chapter{%
  % |\@dottedtocline|の代わりに独自の命令を使う
  \my@dottedtocline
  {1}					%見出しのレベル
  {0.2\baselineskip}		%前の行とのアキ
  {0pt}					%字下げ
  {6\zw}				%ラベル(「第1章」等)の幅
  {\large}				%フォント
%   {\rmfamily}			%フォント
}
\def\l@section{%
  % |\@dottedtocline|の代わりに独自の命令を使う
  \my@dottedtocline
    {2}					%見出しのレベル
    {0.1\baselineskip}	%前の行とのアキ
    {0pt}				%字下げ
    {6\zw}				%ラベル(「第1章」等)の幅
    {\rmfamily\normalsize}	%フォント
}
\def\l@subsection{%
  % |\@dottedtocline|の代わりに独自の命令を使う
  \my@dottedtocline
    {3}					%見出しのレベル
    {0\baselineskip}	%前の行とのアキ
    {0pt}				%字下げ
    {6\zw}				%ラベル(「第1章」等)の幅
    {\rmfamily\normalsize}	%フォント
}
\def\l@subsubsection{%
  % |\@dottedtocline|の代わりに独自の命令を使う
  \my@dottedtocline
    {4}					%見出しのレベル
    {0\baselineskip}	%前の行とのアキ
    {0pt}				%字下げ
    {6\zw}				%ラベル(「第1章」等)の幅
    {\rmfamily\small}			%フォント
}
% jlreqの|\@dottedtocline|を元に独自の命令を定義
\def\my@dottedtocline#1#2#3#4#5#6#7{%
  \jlreq@set@top@contents{#1}%
  \ifnum #1>\c@tocdepth \else
    \vskip #2\relax
    {\leftskip #3\relax \rightskip \@tocrmarg \parfillskip -\rightskip
    \parindent #3\relax\@afterindenttrue
    \interlinepenalty\@M
    \leavevmode
    \@lnumwidth #4\relax
    \@tempcnta=#1\relax
    \advance\@tempcnta by -\jlreq@top@contents
    \@tempdima=1\jlreq@mol
    \multiply \@tempdima by \@tempcnta
    \advance\leftskip \@lnumwidth \hbox{}\hskip -\leftskip
    \advance\leftskip\@tempdima
    {#5#6}\nobreak
    \leaders\hbox{$\m@th\mkern \@dotsep mu$\null\inhibitglue ・\inhibitglue\null$\m@th\mkern \@dotsep mu$}%
    \hfill\nobreak
    \hb@xt@\@pnumwidth{\hss\normalfont\normalcolor #7}%
    \par}%
  \fi}
\makeatother
%----------
%==========
% パッケージ集
%==========
%数式フォント
\usepackage{lxfonts}
\usepackage{bm}%太字のベクトルを表示
\usepackage{mathrsfs}%アルファベット筆記書体など
\DeclareSymbolFont{symbolsC}{U}{txsyc}{m}{n}
\DeclareMathSymbol{\strictif}{\mathrel}{symbolsC}{74}
% 厳密含意 restrict implication の 釣り針矢印 \strictif
%日本語フォント
\usepackage{luatexja-fontspec}
\usepackage[hiragino-pro]{luatexja-preset}
% \setmainjfont{UDDigiKyokashoNP-R}[BoldFont={UDDigiKyokashoNP-B}]
\setmainjfont{UDDigiKyokashoProN-Regular}[BoldFont={UDDigiKyokashoProN-Bold}]
%欧文フォント
% \usepackage[LGR,LGRx,T2A,T1,OT1]{fontenc}
\usepackage[LGR,T2A,T1]{fontenc}
% \usepackage[T1]{fontenc}
% \usepackage[utf8x]{inputenc}
% \usepackage[lutf8]{luainputenc} % lutf8 指定
\usepackage{tgheros}
\usepackage{lmodern}
% \usepackage{pxfonts}
% \usepackage{mathpazo}
% \setmainfont[Scale=MatchLowercase]{Linux Libertine O} % \rmfamily のフォント
% \setsansfont[Scale=MatchLowercase]{Linux Biolinum O}  % \sffamily のフォント
%ギリシャ語フォント
% \usepackage[prefernoncjk]{pxcjkcat} %ギリシャ文字を欧文扱いにする
% \usepackage{pxcjkcat}
% \cjkcategory{sym18}{cjk}
% \usepackage{polyglossia}
\usepackage[polutonikogreek,greek,english,japanese]{babel}
% \languageattribute{greek}{polutoniko}
% ¥usepackage[boldLipsian,10pt,GlyphNames]{teubner}
\usepackage{teubner}
% \usepackage{substitutefont}
% \substitutefont{LGR}{\rmdefault}{porson}
% ¥substitutefont{LGR}{¥rmdefault}{neohellenic}
% 例		¥textgreek{paide'uw}	ギリシャ文字で表示
% 例		¥greeknumeral{2022}		ギリシャ数字で表示
%---------
\usepackage{luatexja-otf}
\usepackage{luatexja-ruby}
\usepackage[unicode,hidelinks,pdfusetitle]{hyperref}
%---------
%\languageattribute{greek}{polutoniko}
%----------
\usepackage{graphicx} %画像の挿入
\usepackage{wrapfig} %図の回り込み
\usepackage{framed,color} %枠付き文書
\definecolor{shadecolor}{gray}{0.8}
% \begin{oframed} 文 \end{oframed}
% \begin{shaded} 文 \end{shaded}
%----------
\usepackage{setspace}
%----------
\usepackage{colortbl}
\usepackage{tikz}
\usepackage{tikz-cd} %可換図式
\usepackage{tikz-3dplot}
\tikzset{cross/.style={preaction={-,draw=white,line width=4pt}}}
\usetikzlibrary{arrows.meta}
% \tikzcdset{
%   arrow style=tikz, % TikZ の arrows.meta を使うモード
%   diagrams={
%     >={Straight Barb[scale=0.8]} % 矢印の先端だけ 0.7 倍に
%   }
% }
\usepackage{pgfplots}
\pgfplotsset{compat=1.12}
%---------
\usepackage{ascmac}
\usepackage{amsmath}
\usepackage{amssymb}
\usepackage{amsthm}
\usepackage{amscd}
\usepackage{pb-diagram} %可換図式
\usepackage{fancybox}
\usepackage{enumerate}
\usepackage{ulem} %下線
\usepackage{mathtools}
%---------
\usepackage{lipsum} %ダミーテキスト 
% 例: \lipsum[1-5]
%---------
%丸囲い
\newcommand*\circled[1]{\tikz[baseline=(char.base)]{
            \node[shape=circle,draw,inner sep=2pt] (char) {#1};}}
\newcommand*\doublecircled[1]{\tikz[baseline=(char.base)]{
           	\node[shape=circle,draw,inner sep=1.6pt] (char) {#1};\node[shape=circle,draw,inner sep=2.4pt] (char) {#1};}}
\newcommand{\doublecirc}{{\ooalign{$\bigcirc$\crcr\hss$\circ$\hss}}}

%---------
\renewcommand{\textgt}[1]{\textsf{\textbf{#1}}}
%---------
\usepackage{multicol} % n段組み 
%\begin{multicols}{段数} 文章 \end{multicols}
%----------
\usepackage{vwcol}
% \begin{vwcol}[widths={0.6,0.4},rule=0.5pt] 
% 文章
% \end{vwcol}
%----------
% \usepackage{physics} %便利なパッケージ
% \qty() , \qty|| ,\norm{} など ...... カッコ
% \vb{} ...... 太字のベクトル
% 数式数式  \qq{テキスト}  数式数式
% \dd[指数]{x} ...... dx
% \dv{x} ............ d/dx
% \pdv{x} ........... ∂/∂x
% \dv{f}{x} ......... df/dx
% \pdv{f}{x} ........ ∂f/∂x
% \mqty(a&b\\c&d) , \mqty|| 行列
% \mqty{\imat{n}} ...... 単位行列
% \mqty{\dmat[0]{n}} ... 対角行列
% \mqty{\xmat*{a}{m}{n}} ... {a_mn}行列
%----------
\usepackage{keyval,physics2}
\usephysicsmodule{ab,ab.braket}
% \ab() , \ab<> , \ab[] カッコ
%----------
\usepackage[version=4]{mhchem} %化学反応式
\usepackage{expl3}
\usepackage{calc}
% \ce{6 CO2 + 12 H2O -> C6H12O6 + 6 O2 + 6 H2O}
\usepackage{listings}
%----------
% Mathematical Logic
\usepackage{turnstile}
% \renewcommand{\vdash}{\sststile{}{}}
% \renewcommand{\vDash}{\sdtstile{}{}}
% \renewcommand{\Vdash}{\dststile{}{}}
\usepackage{bussproofs}
\EnableBpAbbreviations
\renewcommand{\fCenter}{{}\Rightarrow{}}
\renewcommand{\RL}[1]{\RightLabel{{\scriptsize (#1)}}}
%%%%%%%%%% 自然演繹 
% \begin{prooftree}
% 0 \AXC { $ ... $ }  
% 1 \UIC { $ ... $ } 
% 2 \BIC { $ ... $ } 
% 3 \TIC { $ ... $ } 
% \end{prooftree}
%%%%%%%%%% シークエント計算
%\begin{prooftree}
% 0 \AX $ ...\fCenter... $
% 1 \UI $ ...\fCenter... $
% 2 \BI $ ...\fCenter... $
% 3 \TI $ ...\fCenter... $
%\end{prooftree}
%----------
\usepackage{qtree} %タブローパッケージ
%----------
\renewcommand{\labelitemi}{~$\bullet$~}
% \renewcommand{\labelitemii}{~\circ~}
\renewcommand{\labelitemii}{~$\triangleright$~}
% \renewcommand{\labelitemii}{~$\bullet$~} %
% \renewcommand{\labelitemiii}{~$\bullet$~} %
%----------
\usepackage{emoji} %絵文字
% \setemojifont{EmojiOneMozilla}
% \setemojifont{Noto Emoji Regular}
%----------
\newtheorem{definition}{定義}%[section]
\newtheorem{proposition}[definition]{命題}
\newtheorem{theorem}[definition]{定理}
\newtheorem{lemma}[definition]{補題}
\newtheorem{corollary}[definition]{系}
\newtheorem{example}[definition]{例}
\newtheorem{practice}[definition]{演習問題}
% \newtheorem*{longproof}{証明}
\newtheorem*{answer}{解答}
\newtheorem*{supplement}{補足}
\newtheorem*{remark}{注意}
%----------
% 定理環境(tcolorbox)
\usepackage{tcolorbox} %箱
\tcbuselibrary{breakable,skins,theorems}
%----------
\tcolorboxenvironment{definition}{
	blanker,breakable,
	left=3mm,right=3mm,
	top=3mm,bottom=3mm,
	before skip=15pt,after skip=15pt,
	borderline={0.5pt}{0pt}{black}
}
\newtcolorbox{emptydefinition}{
	blanker,breakable,
	left=3mm,right=3mm,
	top=3mm,bottom=3mm,
	before skip=15pt,after skip=15pt,
	borderline={0.5pt}{0pt}{black}
}
%----------
\tcolorboxenvironment{proposition}{
	blanker,breakable,
	left=3mm,right=3mm,
	top=3mm,bottom=3mm,
	before skip=15pt,after skip=15pt,
	borderline={0.5pt}{0pt}{black}
}
\newtcolorbox{emptyproposition}{
	blanker,breakable,
	left=3mm,right=3mm,
	top=3mm,bottom=3mm,
	before skip=15pt,after skip=15pt,
	borderline={0.5pt}{0pt}{black}
}
%----------
\tcolorboxenvironment{theorem}{
	blanker,breakable,
	left=3mm,right=3mm,
	top=3mm,bottom=3mm,
    sharp corners,boxrule=0.6pt,
	before skip=15pt,after skip=15pt,
	borderline={0.5pt}{0pt}{black},
    borderline={0.5pt}{1.5pt}{black}
}
\newtcolorbox{emptytheorem}{
	blanker,breakable,
	left=3mm,right=3mm,
	top=3mm,bottom=3mm,
    sharp corners,boxrule=0.6pt,
	before skip=15pt,after skip=15pt,
	borderline={0.5pt}{0pt}{black},
    borderline={0.5pt}{1.5pt}{black}
}
%----------
\tcolorboxenvironment{lemma}{
	blanker,breakable,
	left=3mm,right=3mm,
	top=3mm,bottom=3mm,
	before skip=15pt,after skip=15pt,
	borderline={0.5pt}{0pt}{black}
}
%----------
\tcolorboxenvironment{corollary}{
	blanker,breakable,
	left=3mm,right=3mm,
	top=3mm,bottom=3mm,
	before skip=15pt,after skip=15pt,
	borderline={1.0pt}{0pt}{black,dotted}
}
\newtcolorbox{emptycorollary}{
	blanker,breakable,
	left=3mm,right=3mm,
	top=3mm,bottom=3mm,
	before skip=15pt,after skip=15pt,
	borderline={1.0pt}{0pt}{black,dotted}
}
%----------
\tcolorboxenvironment{example}{
	blanker,breakable,
	left=3mm,right=3mm,
	top=3mm,bottom=3mm,
	before skip=15pt,after skip=15pt,
	borderline={0.5pt}{0pt}{black}
}
%----------
\tcolorboxenvironment{practice}{
	blanker,breakable,
	left=3mm,right=3mm,
	top=3mm,bottom=3mm,
	before skip=15pt,after skip=15pt,
	borderline={0.5pt}{0pt}{black}
}
%----------
\tcolorboxenvironment{proof}{
	blanker,breakable,
	left=3mm,right=3mm,
	top=2mm,bottom=2mm,
	before skip=15pt,after skip=20pt,
	% borderline west={1.5pt}{0pt}{black,dotted}
	borderline vertical={1pt}{0pt}{black,dotted}
	% borderline vertical={0.8pt}{0pt}{black,dotted,arrows={Square[scale=0.5]-Square[scale=0.5]}}
	}
%----------
\tcolorboxenvironment{supplement}{
	blanker,breakable,
	left=3mm,right=3mm,
	top=2mm,bottom=2mm,
	before skip=15pt,after skip=20pt,
	% borderline west={1.5pt}{0pt}{black,dotted}
	% borderline vertical={0.5pt}{0pt}{black,arrows = {Circle[scale=0.7]-Circle[scale=0.7]}}
	borderline vertical={0.5pt}{0pt}{black}
	% borderline vertical={0.5pt}{0pt}{black},
	% borderline north={0.5pt}{0pt}{white,arrows={Circle[black,scale=0.7]-Circle[black,scale=0.7]}}
	}
%----------
\tcolorboxenvironment{remark}{
	blanker,breakable,
	left=3mm,right=3mm,
	top=1mm,bottom=1mm,
	before skip=15pt,after skip=20pt,
	% borderline west={1.5pt}{0pt}{black,dotted}
	% borderline vertical={0.5pt}{0pt}{black,arrows = {Circle[scale=0.7]-Circle[scale=0.7]}}
	borderline vertical={0.5pt}{0pt}{black}
	% borderline vertical={0.5pt}{0pt}{black},
	% borderline north={0.5pt}{0pt}{white,arrows={Circle[black,scale=0.7]-Circle[black,scale=0.7]}}
	}
%----------
% マークシート記号
\newcommand{\egg}[1]{\raisebox{-3pt}{
	\begin{tikzpicture}[x=1pt,y=1pt,line width=1pt]
		\draw (0,0) ellipse (4.5 and 6);
		\draw (0,0) node {
			\usefont{T1}{phv}{m}{n}
			\fontsize{9pt}{0}\selectfont #1 \/};
	\end{tikzpicture}}}
% マークシート記号
\newcommand{\eggg}[1]{\raisebox{-3pt}{
	\begin{tikzpicture}[x=1pt,y=1pt,line width=1pt]
		\draw[fill=black!30] (0,0) ellipse (4.5 and 6);
		\draw (0,0) node {
			\usefont{T1}{phv}{m}{n}
			\fontsize{9pt}{0}\selectfont #1 \/};
	\end{tikzpicture}}}
%-------------------- 
% UDカラー (アクセントカラー)
\definecolor{accentred}{rgb}{1.0, 0.3, 0.0}
\definecolor{accentyellow}{rgb}{1.0, 0.9, 0}
\definecolor{accentgreen}{rgb}{0.0, 0.7, 0.5}
\definecolor{accentblue}{rgb}{0.0, 0.35, 1.0}
\definecolor{accentsky}{rgb}{0.3, 0.76, 1.0}
\definecolor{accentpink}{rgb}{1.0, 0.5, 0.5}
\definecolor{accentorange}{rgb}{0.96, 0.67, 0.0}
\definecolor{accentpurple}{rgb}{0.6, 0.0, 0.6}
%--------------------
% UDカラー (ベースカラー)
\definecolor{basepink}{rgb}{1.0, 0.8, 0.75}
\definecolor{basecream}{rgb}{1, 1, 0.5}
\definecolor{baseyellowgreen}{rgb}{0.8, 0.94, 0.33}
\definecolor{basesky}{rgb}{0.75, 0.9, 1.0}
\definecolor{basebeige}{rgb}{1.0, 0.8, 0.5}
\definecolor{basegreen}{rgb}{0.47, 0.85, 0.65}
\definecolor{basepurple}{rgb}{0.8, 0.67, 0.9}
%--------------------
\definecolor{basegray}{rgb}{0.784,0.784,0.796}
%--------------------
% \pagestyle{empty}  %% ページ番号を消す
% \renewcommand{¥qedsymbol}{¥parallel}
%--------------------
\newcommand{\AP}{\ensuremath{\mathrm{AP}}}
\newcommand{\id}{\ensuremath{\mathrm{id}}}
\newcommand{\dom}{\ensuremath{\mathrm{dom}}}
\newcommand{\codom}{\ensuremath{\mathrm{codom}}}
\newcommand{\Stream}{\ensuremath{\mathrm{Stream}}}
\newcommand{\head}{\ensuremath{\mathsf{head}}}
\newcommand{\tail}{\ensuremath{\mathsf{tail}}}
\newcommand{\even}{\ensuremath{\mathsf{even}}}
\newcommand{\odd}{\ensuremath{\mathsf{odd}}}
\newcommand{\zip}{\ensuremath{\mathsf{zip}}}
\newcommand{\nex}{\ensuremath{\mathsf{next}}}
\newcommand{\beh}{\ensuremath{\mathrm{beh}}}
\newcommand{\bisim}{\ensuremath{\mathbin{\,\underline{\!\leftrightarrow\!}\,}}}
\newcommand{\Obj}{\ensuremath{\mathrm{Obj}}}
\newcommand{\Arr}{\ensuremath{\mathrm{Arr}}}
\newcommand{\Set}{\ensuremath{\mathbf{Set}}}
\newcommand{\Sets}{\ensuremath{\mathbf{Sets}}}
\newcommand{\Rel}{\ensuremath{\mathbf{Rel}}}
\newcommand{\ERel}{\ensuremath{\mathbf{ERel}}}
\newcommand{\EnRel}{\ensuremath{\mathbf{EnRel}}}
\newcommand{\Grp}{\ensuremath{\mathbf{Grp}}}
\newcommand{\Mon}{\ensuremath{\mathbf{Mon}}}
\newcommand{\Cat}{\ensuremath{\mathbf{Cat}}}
\newcommand{\Pred}{\ensuremath{\mathbf{Pred}}}
% \newcommand{\colim}{\ensuremath{\mathrm{colim}}}
\DeclareMathOperator*{\colim}{colim}
\newcommand{\nil}{\ensuremath{\mathsf{nil}}}
\newcommand{\cons}{\ensuremath{\mathsf{cons}}}
\newcommand{\len}{\ensuremath{\mathsf{len}}}
\newcommand{\initial}{\ensuremath{\mathsf{initial}}}
\newcommand{\EMAlg}{\ensuremath{\mathrm{EMAlg}}}
\newcommand{\Coalg}{\ensuremath{\mathrm{Coalg}}}
%--------------------
\renewcommand\A{\mathbb{A}}
\renewcommand\B{\mathbb{B}}
\renewcommand\C{\mathbb{C}}
\renewcommand\D{\mathbb{D}}
\newcommand\E{\mathbb{E}}
\renewcommand\F{\mathbb{F}}
\renewcommand\G{\mathbb{G}}
% \newcommand\H{\mathbb{H}}
\newcommand\I{\mathbb{I}}
\newcommand\J{\mathbb{J}}
\newcommand\K{\mathbb{K}}
% \newcommand\L{\mathbb{L}}
\renewcommand\M{\mathbb{M}}
\newcommand\N{\mathbb{N}}
% \newcommand\O{\mathbb{O}}
% \newcommand\P{\mathbb{P}}
\newcommand\Q{\mathbb{Q}}
\newcommand\R{\mathbb{R}}
% \newcommand\S{\mathbb{S}}
\newcommand\T{\mathbb{T}}
\renewcommand\U{\mathbb{U}}
\newcommand\V{\mathbb{V}}
\newcommand\W{\mathbb{W}}
\renewcommand\X{\mathbb{X}}
\newcommand\Y{\mathbb{Y}}
\newcommand\Z{\mathbb{Z}}
% %--------------------
\renewcommand{\bra}[1]{\langle {#1}\rvert}
\renewcommand{\ket}[1]{\lvert {#1} \rangle}
\renewcommand{\braket}[1]{\langle {#1} \rangle}
\newcommand{\Bra}[1]{\left\langle {#1} \right\rvert}
\newcommand{\Ket}[1]{\left\lvert {#1} \right\rangle}
\newcommand{\Braket}[1]{\left\langle {#1} \right\rangle}
%-----
\newcommand{\computationrightarrow}{\ensuremath{\mathrel{\;{\shortmid}\hspace{-12pt}\xrightarrow{\;~\;}}}}
\newcommand{\computationlongrightarrow}{\ensuremath{\mathrel{~\,{\shortmid}\hspace{-14pt}\longrightarrow}}}
\newcommand{\computationxrightarrow}[1]{\ensuremath{\overset{\substack{#1\\[-3pt]}\;}{\computationlongrightarrow}}}